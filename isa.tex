% Copyright (C) 2014 Sol Boucher and Evan Klei
% This is free software: you can redistribute it and/or modify
% it under the terms of the GNU General Public License as published by
% the Free Software Foundation, either version 3 of the License, or
% (at your option) any later version.
%
% This is distributed in the hope that it will be useful,
% but WITHOUT ANY WARRANTY; without even the implied warranty of
% MERCHANTABILITY or FITNESS FOR A PARTICULAR PURPOSE.  See the
% GNU General Public License for more details.
%
% You should have received a copy of the GNU General Public License
% along with it.  If not, see <http://www.gnu.org/licenses/>.

\documentclass[12pt]{article}
\usepackage{framed}
\usepackage[margin=1in,top=0.5in]{geometry}
\usepackage{multirow}
\thispagestyle{empty}
\renewcommand{\thefootnote}{\fnsymbol{footnote}}
\begin{document}
\setlength{\parindent}{0pt}

\begin{framed}
Sol Boucher and Evan Klei \hfill CSCI-453-01 \hfill 04/09/14 \\
\vspace{6pt} \\
\centerline{\textbf{\huge FabComp: ISA document}}
\end{framed}

\section{Overview}
Our Fabulous Computation Machine, or FabComp for short, is a 16-bit general-purpose register--based, big-endian CISC architecture.
It proudly features an extensive collection of keystroke-saving instructions, 16 general-purpose registers, and 9 memory address modes.
Instructions use hybrid encoding, with the opcode word's format fixed, but between 0 and 6 trailing immediate words.
Comparisons are done using conditional registers, but we also provide conditional-and-branch instructions for convenience.

\section{Instruction format}

\subsection{Opcode word}
Every instruction begins with an opcode word that always follows this format:

\vspace{6pt}
\begin{tabular}{| c | c | c | c | c |}
\hline
\multicolumn{5}{| c |}{opcode word format} \\
\hline
opcode & immediate & am0 & am1 & am2 \\
\hline
15 \hfill 8 & 7 \hfill 6 & 5 \hfill 4 & 3 \hfill 2 & 1 \hfill 0 \\
\end{tabular}
\begin{tabular}{| r | l |}
\hline
\multicolumn{2}{| c |}{meaning of immediate flags} \\
\hline
value & meaning \\
\hline
00 & no operands are immediates \\
01 & operand 2 is an immediate value \\
10 & operand 1 is an immediate value \\
11 & operands 1 and 2 are immediates \\
\hline
\end{tabular}

\vspace{6pt}
The immediate flags can override the address modes of operands 1 and 2: if set, the corresponding operand will be an immediate value rather than a register or memory address.
Otherwise, the corresponding \texttt{am} (address mode) bits are consulted:

\vspace{6pt}
\begin{tabular}{| c | l | c |}
\hline
\multicolumn{3}{| c |}{meaning of address mode flags} \\
\hline
value & meaning & first immediate instruction word \\
\hline
00 & skip: No operand & None \\
01 & immediate address: Memory address & I-type\\
10 & PC-relative: Offset from program counter & I-type \\
11 & other: Immediate word will specify address mode & R-type \\
\hline
\end{tabular}

\subsection{Immediate words}
There are three types of immediate words:

\vspace{6pt}
\begin{tabular}{| c |}
\hline
I-type immediate word format \\
\hline
immediate \\
\hline
15 \hfill 0
\end{tabular}
\begin{tabular}{| c | c |}
\hline
\multicolumn{2}{| c |}{S-type immediate word format} \\
\hline
short immediate 0 & short immediate 1 \\
\hline
15 \hfill 8 & 7 \hfill 0 \\
\end{tabular}

\vspace{6pt}
\begin{tabular}{| c | c | c | c |}
\hline
\multicolumn{4}{| c |}{R-type immediate word format} \\
\hline
secondary am & reg0 & reg1 & reg2 \\
\hline
15 \hfill 12 & 11 \hfill 8 & 7 \hfill 4 & 3 \hfill 0 \\
\end{tabular}

\vspace{6pt}
\begin{tabular}{| c | l | c |}
\hline
\multicolumn{3}{| c |}{meaning of secondary address mode flags} \\
\hline
value & meaning & next immediate word(s) \\
\hline
000 & Register value: reg0 specifies a register & None \\
001 & Register indirect: Find EA in register reg0 & None \\
010 & Scaled: EA calculated as offset from reg0 & S-type \\
011 & Doubly scaled: For 2-dimensional arrays with based in register & S-type \\
100 & Auto increment: Find EA in register reg0, then increment & None \\
101 & Auto decrement: Find EA in register reg0, then decrement & None \\
110 & Scaled displacement: EA calculated as offset from immedate & S-type then I-type \\
111 & Doubly scaled displacment: 2-dimensional immediate-based & S-type then I-type \\
\hline
\end{tabular}

\section{Instruction set}
We support the following instructions, all of which are encoded with a standard opcode word:
\subsection{List of instructions}
\vspace{6pt}
\begin{tabular}{| r | r | c | l |}
\hline
& \textbf{opcode} & \textbf{mnemonic} & \textbf{description} \\
\hline
0 & 00000000 & HALT & Stop running \\
\hline
\multirow{23}{*}{1} & 00000001 & AND & Perform bitwise and \\
& 00000010 & OR & Perform bitwise or \\
& 00000011 & XOR & Perform bitwise xor \\
& 00000100 & LSFT & Perform bitwise left shift\\
& 00000101 & NAND & Perform bitwise nand\\
& 00000110 & NOR & Perform bitwise nor\\
& 00000111 & XNOR & Perform bitwise xnor\\
& 00001000 & RSFT & Perform bitwise right shift\\
& 00001001 & LAND & Performs logical and \\
& 00001010 & LOR & Performs logical or\\
& 00001011 & LXOR & Performs logical xor\\
& 00001100 & RASFT & Perform bitwise right arithmatic shift\\
& 00001101 & LNAND & Performs logical nand\\
& 00001110 & LNOR & Performs logical nor\\
& 00001111 & LXNOR & Performs logical xnor\\
& 00010000 & SLT & Compares if one object is less than another\\
& 00010001 & SGT & Compares if one object is greater than another\\
& 00010010 & SEQ & Compares if two objects are equal\\
& 00010011 & SNE & Compares if two objects are not equal\\
& 00010100 & SLE & Compares if one object is less than or equal to another\\
& 00010101 & SGE & Compares if one object is greater than or equal to another\\
& 00010110 & ADD & Perform signed arithmetic addition \\
& 00010111 & SUB & Perform signed arithmetic subtraction \\
\hline
\multirow{6}{*}{2} & 00011000 & BLT & Branches if one object is less than the other\\
& 00011001 & BGT & Branches if one object is greater than the other\\
& 00011010 & BEQ & Branches if two objects are equal\\
& 00011011 & BNE & Branches if two objects aren't equal\\
& 00011100 & BLE & Branches if one object is less than or equal to the other\\
& 00011101 & BGE & Branches if one object is greater than or equal to the other\\
\hline
3 & 00011110 & PRNT & Prints the value at the specified location \\
\hline
\multirow{4}{*}{4} & 00011111 & LNOT & Performs a logical not\\
& 00100000 & SIZ & Compares if the object is zero\\
& 00100001 & SNZ & Compares if the object is not zero\\
& 00100010 & NOT & Performs a bitwise not\\
& 00100011 & NEG & Performs a arithmetic negation\\
\hline
\multirow{2}{*}{5} & 00100100 & BIZ & Branch if the object is zero\\
& 00100101 & BNZ & Branch if the object is not zero\\
\hline
\multirow{2}{*}{6} & 00100110 & INCR & Increments the data in the given location\\
& 00100111 & DECR & Decriments the data in the given location\\
\hline
\multirow{3}{*}{7} & 00101000 & JMP & Jumps to the location given\\
& 00101001 & JAL & Jumps to the location given and sets the RA\\
& 00101010 & CALL & Jumps to the location and puts the address on the stack\\
\hline
8 & 00101011 & RET & Jumps to the location most recently put on the stack\\
\hline
9 & 00101100 & MOV & Moves object from one space to another\\
\hline
\end{tabular}

\subsection{Verification rules}
Each numbered grouping in the instruction set table is validated as follows:

\begin{enumerate}
\setcounter{enumi}{-1}
\item \begin{itemize} \item No validation \end{itemize}
\item \begin{itemize} \item 2-3 ops \end{itemize}
\item \begin{itemize}
    \item 3 ops
    \item op0 is not a register
    \end{itemize}
\item \begin {itemize} \item 1 op\footnotemark \end{itemize}
\item \begin {itemize} \item 1-2 ops \end{itemize}
\item \begin {itemize}
     \item 2 ops
     \item op0 is not a register
     \end{itemize}
\item \begin{itemize} \item 1 op \end{itemize}
\item \begin{itemize}
        \item 1 op
        \item op0 is not a register
      \end{itemize}
\item \begin {itemize} \item 0 ops \end{itemize}
\item \begin {itemize} \item 2 ops \end{itemize}
\end{enumerate}

\footnotetext{Although the operand is passed in position 0, it must be treated as a source operand}

\section{Memory specifications}
The memory space is split into 16 bit-words, arranged big-endian.
Each location is identified via a 16-bit address, which points to a whole word.
Because addresses are assigned to words instead of bytes, it impossible to have unassigned accesses.
The bus to and from memory is 16 bits, and the ISA only supports transferring a single word at a time.

\section{List of registers}
The architecture exposes the following 16 general-purpose registers:

\vspace{6pt}
\begin{tabular}{| r | l |}
\hline
\textbf{Register} & \textbf{Description}\\
\hline
\$a0--\$a2 & Argument Registers\\
\$v & Return data Register\\
\$s0--\$s3 & Saved Registers\\
\$t0--\$t5 & Temporary Registers\\
\$sp & Stack Pointer\\
\$ra & Return Address\\ 
\hline
\end {tabular}

\section{Address modes and formats}
\subsection{Calculating Address Modes}
\begin{tabular}{| r | l |}
\hline
\textbf{Address Mode} & \textbf {Effective Address} \\
\hline
Immediate Address & imm\\
PC Relative & EA = \$pc \\
Register Indirect & EA = \$reg0\\
Scaled & EA = \$reg0 + \$reg1 $<<$ imm0\\
Doubly Scaled & EA = Mem[\$reg0 + \$reg1 $<<$ imm0] + \$reg2 $<<$ imm1 \\
Auto Increment & EA = \$reg0 , \$reg0 = \$reg0 + 1\\
Auto Decrement & EA = \$reg0 , \$reg0 = \$reg0 -- 1\\
Scaled Displacement & EA = imm + \$reg0 $<<$ imm0\\
Doubly Scaled Displacemnt & EA = Mem[imm + \$reg0 $<<$ imm0] + \$reg1 $<<$ imm1\\
\hline
\end{tabular}

\end{document}
