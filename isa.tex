\documentclass[12pt]{article}
\usepackage{framed}
\usepackage[margin=1in]{geometry}
\usepackage{multirow}
\thispagestyle{empty}
\begin{document}

\begin{framed}
\noindent
Sol Boucher and Evan Klei \hfill CSCI-453-01 \hfill 04/09/14 \\
\vspace{6pt} \\
\centerline{\textbf{\huge FabComp: ISA document}}
\end{framed}

\section{Overview}
Our Fabulous Computation Machine, or FabComp for short, is a 16-bit general-purpose register--based, big-endian CISC architecture.
It proudly features an extensive collection of keystroke-saving instructions, 16 general-purpose registers, and 9 memory address modes.
Instructions use hybrid encoding, with the opcode word's format fixed, but between 0 and 6 trailing immediate words.
Comparisons are done using conditional registers, but we also provide conditional-and-branch instructions for convenience.

\section{Instruction format}

\section{List of instructions}
We support the following instructions, all of which are encoded with a standard opcode word:
\subsection{List of instructions}
\vspace{6pt}
\noindent
\begin{tabular}{| r | r | c | l |}
\hline
& \textbf{opcode} & \textbf{mnemonic} & \textbf{description} \\
\hline
0 & 00000000 & HALT & Stop running \\
\hline
\multirow{23}{*}{1} & 00000001 & AND & Perform bitwise and \\
& 00000010 & OR  & Perform bitwise or \\
& 00000011 & XOR & Perform bitwise xor \\
& 00000100 & LSFT & Perform bitwise left shift\\
& 00000101 & NAND & Perform bitwise nand\\
& 00000110 & NOR & Perform bitwise nor\\
& 00000111 & XNOR & Perform bitwise xnor\\
& 00001000 & RSFT & Perform bitwise right shift\\
& 00001001 & LAND & Performs logical and \\
& 00001010 & LOR & Performs logical or\\
& 00001011 & LXOR & Performs logical xor\\
& 00001100 & RASFT & Perform bitwise right arithmatic shift\\
& 00001101 & LNAND & Performs logical nand\\
& 00001110 & LNOR & Performs logical nor\\
& 00001111 & LXNOR & Performs logical xnor\\
& 00010000 & SLT & Compares if one object is less than another\\
& 00010001 & SGT & Compares if one object is greater than another\\
& 00010010 & SEQ & Compares if two objects are equal\\
& 00010011 & SNE & Compares if two objects are not equal\\
& 00010100 & SLE & Compares if one object is less than or equal to another\\
& 00010101 & SGE & Compares if one object is greater than or equal to another\\
& 00010110 & ADD & Perform signed arithmetic addition \\
& 00010111 & SUB & Perform signed arithmetic subtraction \\
\hline
\multirow{6}{*}{2} & 00011000 & BLT & Branches if one object is less than the other\\
& 00011001 & BGT & Branches if one object is greater than the other\\
& 00011010 & BEQ & Branches if two objects are equal\\
& 00011011 & BNE & Branches if two objects aren't equal\\
& 00011100 & BLE & Branches if one object is less than or equal to the other\\
& 00011101 & BGE & Branches if one object is greater than or equal to the other\\
\hline
\multirow{2}{*}{3} & 00011110 & &\\
& 00011111 & &\\
\hline
\multirow{4}{*}{4} & 00100000 & SIZ & Compares if the object is zero\\
& 00100001 & SNZ & Compares if the object is not zero\\
& 00100010 & NOT & Performs a bitwise not\\
& 00100011 & NEG & Performs a arithmetic negation\\
\hline
\multirow{2}{*}{5} & 00100100 & BIZ & Branch if the object is zero\\
& 00100101 & BNZ & Branch if the object is not zero\\
\hline
6 & 00100110 & MOV & Moves object from one space to another\\
\hline
7 & 00100111 & RET & Jumps to the location most recently put on the stack\\
\hline
\multirow{3}{*}{8} & 00101000 & JMP & Jumps to the location given\\
& 00101001 & JAL & Jumps to the location given and sets the RA\\
& 00101010 & CALL & Jumps to the location and puts the address on the stack\\
\hline
\end{tabular}
\subsection{Verification rules}
\begin{enumerate}
\setcounter{enumi}{-1}
\item \begin{itemize} \item no validation \end{itemize}    
\item \begin{itemize} \item 2-3 ops \end{itemize}        
\item \begin{itemize}
    \item 3 ops
    \item op0 is neither immediate nor register        
    \end{itemize}
\item \begin {itemize} \item Fails validation \end{itemize}  
\item \begin {itemize} \item 1-2 ops \end{itemize}
\item \begin {itemize}
     \item 2 ops
     \item op0 is neither immediate nor register
     \end{itemize}
\item \begin {itemize} \item 2 ops \end{itemize}
\item \begin {itemize} \item 0 ops \end{itemize}
\item \begin{itemize}
        \item 1 ops
        \item op0 is neither immediate nor register
      \end{itemize}
\end{enumerate}
\section{Memory specifications}
The memory space is composed 16 bit words, arranged big-endian.
Each place is identified via a 16 bit address, which points to a 16 bit unit.
The bus to and from memory is 16 bits as well.
\section{List of registers}
\begin{tabular}{| r | l |}
\hline
\textbf{Register} & \textbf{Description}\\
\hline
\$a0-\$a2 & Argument Registers\\
\$v & Return data Register\\
\$s0-\$s3 & Saved Registers\\
\$t0-\$t5 & Temporary Registers\\
\$sp & Stack Pointer\\
\$ra & Return Address\\ 
\hline
\end {tabular}

\section{Address modes and formats}

\end{document}
