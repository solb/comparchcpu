\documentclass[12pt]{article}
\usepackage{framed}
\usepackage[margin=1in]{geometry}
\thispagestyle{empty}
\begin{document}

\begin{framed}
\noindent
Sol Boucher and Evan Klei \hfill CSCI-453-01 \hfill 04/09/14 \\
\vspace{6pt} \\
\centerline{\textbf{\huge FabComp: ISA document}}
\end{framed}

\section{Overview}
Our Fabulous Computation Machine, or FabComp for short, is a 16-bit general-purpose register--based, big-endian CISC architecture.
It proudly features an extensive collection of keystroke-saving instructions, 16 general-purpose registers, and 9 memory address modes.
Instructions use hybrid encoding, with the opcode word's format fixed, but between 0 and 6 trailing immediate words.
Comparisons are done using conditional registers, but we also provide conditional-and-branch instructions for convenience.

\section{Instruction format}

\section{List of instructions}
We support the following instructions, all of which are encoded with a standard opcode word:

\vspace{6pt}
\begin{tabular}{| r | c | l |}
\hline
\textbf{opcode} & \textbf{mnemonic} & \textbf{description} \\
\hline
00000000 & ADD & Perform signed arithmetic addition \\
00000001 & SUB & Perform signed arithmetic subtraction \\
\hline
\end{tabular}

\section{Memory specifications}

\section{List of registers}

\section{Address modes and formats}

\end{document}
